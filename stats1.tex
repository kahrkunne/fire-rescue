\documentclass{article}
\begin{document}
\title{Statistics Homework 1}
\author{Kahr Kunne (S3435342)}
\date{2019.09.20}

\maketitle

\begin{enumerate}
\item
  \begin{enumerate}
  \item Half of values are lower than the median. If the mean is also lower
    than the median, then less than half of values are lower than the mean,
    i.e. there are more ``high'' values than ``low'' values. We expect the
    distribution to be left-skewed.
  \item \begin{itemize}
    \item quantile: a cut-off point we use to divide a probability
      distribution into regions of equal probability.
    \item P-value: the probability that if the null hypothesis is true,
      statistical results derived from the population would be at least
      as extreme as our actual observed results.
    \item Type-I error: when we reject a true null hypothesis.
    \end{itemize}
  \item A confidence interval is a range which we are reasonably confident
    contains the true mean of the population, with the definition of
    ``reasonably confident'' depending on our chosen confidence level.
  \end{enumerate}

\item
  \begin{enumerate}
    \item Since it concerns a large number of independent, random events, we
      expect a distribution that is approximately normal. We expect a mean of
      $10^{-5} \cdot 10^6 = 10$ and a variance of $...$.
      \item 
  \end{enumerate}



\end{enumerate}

\end{document}
